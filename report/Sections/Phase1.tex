\section{Phase 1}

\subsection{Encoder}

\label{sec:encoder}

This section details the two source-coding strategies \emph{Huffman coding} and a plain \emph{ASCII}. All scripts were written in \textsc{Matlab}.

\subsubsection{Huffman encoder}
\label{ssec:huffman}


Firstly the number of bits in the bit-stream produced shall be divisible by 4, this is because the message will be sent with 16-QAM and each symbol has 4 bits. By doing this in this stage, the results for all modulators will be comparable, since they will be sending the same message. The padding was implemented by adding spaces at the end of the message until the huffman encoded message was divisible by 4.

The message, once padding has been applied, appears as follows:

\begin{center}
    \begin{verbatim}
        "polar codes are employed in 5g due better performance and simplicity   "
    \end{verbatim}
    
\end{center}

For the padded test sentence (71 symbols) the huffman alphabet is described by Table \ref{tab:huffCode}.

\begin{table}[H]
    \centering
    \caption{Message Huffman Coding}
    \begin{tabularx}{\textwidth}{>{\centering\arraybackslash}X >{\centering\arraybackslash}X >{\centering\arraybackslash}X >{\centering\arraybackslash}X}
        \toprule
        \textbf{Character} & \textbf{Probability} & \textbf{Code} & \textbf{Code Length} \\
        \midrule
        ' ' & 0.183099 & 11 & 2\\
        \midrule
        '5' & 0.014085 & 0000011 & 7\\
        \midrule
        'a' & 0.056338 & 0111 & 4\\
        \midrule
        'b' & 0.014085 & 0110001 & 7\\
        \midrule
        'c' & 0.042254 & 00100 & 5\\
        \midrule
        'd' & 0.056338 & 1001 & 4\\
        \midrule
        'e' & 0.126761 & 010 & 3\\
        \midrule
        'f' & 0.014085 & 011001 & 6\\
        \midrule
        'g' & 0.014085 & 0110000 & 7\\
        \midrule
        'i' & 0.056338 & 1010 & 4\\
        \midrule
        'l' & 0.042254 & 00101 & 5\\
        \midrule
        'm' & 0.042254 & 00010 & 5\\
        \midrule
        'n' & 0.042254 & 00001 & 5\\
        \midrule
        'o' & 0.056338 & 1011 & 4\\
        \midrule
        'p' & 0.056338 & 1000 & 4\\
        \midrule
        'r' & 0.070423 & 0011 & 4\\
        \midrule
        's' & 0.028169 & 01101 & 5\\
        \midrule
        't' & 0.042254 & 00011 & 5\\
        \midrule
        'u' & 0.014085 & 0000010 & 7\\
        \midrule
        'y' & 0.028169 & 000000 & 6\\
        \bottomrule
    \end{tabularx}
    \label{tab:huffCode}
\end{table}

This message has the following characteristics:

\begin{itemize}
  \item bitstream length $\lvert c\rvert = 284$ bits,
  \item average codeword length $\bar L = 4.00$ bits/symbol,
  \item Entropy $H(X)=3.95$ bits/symbol
\end{itemize}


\subsection{ASCII reference}
\label{ssec:ascii}

The benchmark encodes each character using its standard 8-bit ASCII
representation (\lstinline|uint8| $\rightarrow$ \lstinline|de2bi|).  No padding
is required because every symbol already expands to a full byte.

\vspace{0.5em}
\noindent
\textbf{Results} for the \emph{same} padded sentence:
\begin{itemize}
  \item bitstream length $\lvert c\rvert = 544$ bits,
  \item average codeword length $\bar L = 8$ bits/symbol,
  \item coding efficiency $H(X)/\bar L = 49.4\,\%$,
  \item compression gain vs.\ Huffman: $544/284 \approx 1.92\times$ fewer bits.
\end{itemize}


\subsection{Scalability for channel simulation}

Both encoders must supply at least $10^6$ bits to the modulator so that
\textsc{BER}/\textsc{FER} curves exhibit the desired confidence
intervals.  The scripts therefore replicate the base sequence until the
threshold is reached:

The procedure preserves the symbol statistics and avoids edge effects
during convolutional/concatenated decoding.

\subsection{Take-away} 
Huffman coding achieves a near-ideal $99\%$ source-coding efficiency,
reducing the payload by almost a factor of two relative to the naïve
ASCII baseline.  This gain translates directly into higher spectral
efficiency or, equivalently, a $3$ dB signal-to-noise margin when the
system operates at a fixed code rate and modulation order.