\section{Phase II}
\label{sec:phase2}




\subsection{Resultados}

3 Figuras com 4 curvas, conv bloco e seguido, curvas comb canais e modulacoes.

Valor teorico das probabilidades de retransmissao e de erro plot

tabela ErrorRate | Numero de Erros | 




Figure~\ref{fig:RxTx} shows the complete Block Diagram. For this second phase 2 blocks are added, Interleaver and error control. 

\subsection{Interleaver}

The \textbf{Interleaver} will shuffle the bit order, because errors tend to happen in bursts, this way the code correction/detection block is more efficient. Hence, the interleaver order is of importance, it must be after error control, otherwise it will be useless, and less obviously it must be before Modulation, because each modulated symbol is represents more than 1 bit, therefore an interleaver for the modulated symbols will have bit error bursts. For the simulation, this block should be more effective for Rayleigh because it has more bursts, compared to AWGN where it is just white noise. 

\subsection{Error Control}

For both error-control schemes the MATLAB processing chain is identical: the plain-text message is encoded, protected by the selected error control stage, interleaved, and modulated. After transmission through the channel, the signal is demodulated, de-interleaved, and decoded in reverse order, and the resulting bits are compared with the original data to compute the error statistics.

\subsubsection{Convolutional}

The convolutional coding uses 64 states and has a code rate $R = 1/3$, constraint length $K = 7$ and generator polinomial $G_0 = 133$, $G_1=171$ and $G_3 = 165$. 


Dar plot aos valores teoricos e mostrar as equacoes.

\subsubsection{Hamming}

The other block-coding stage employed is the binary Hamming $(15,11)$ code. It maps every $k=11$ information bits into an $n=15-bit$ codeword, adding four parity bits according to the generator matrix, The code can correct any single-bit error and detect (but not correct) any double-bit error in each 15-bit word.

\begin{comment}

        \begin{itemize}
          \item \textit{Convolutional} encoder, rate~$1/3$, constraint
                length $K=7$; generator polynomials in octal  
                $G_0 = 133,\; G_1 = 171,\; G_2 = 165$  
                (see Table 9-2 in \textit{TAC Topics}).  The
                Viterbi algorithm is used for decoding.
          \item \textit{Hamming} $(15,\,11)$ block code; generator
                matrix  
                \(
                  \mathbf G =
                    \bigl[I_{11}\;|\;\mathbf P_{11\times4}\bigr]
                \)
                (full matrix in App.~\ref{app:hamming}).
          \item \textit{Concatenated} scheme: Hamming outer +
                convolutional inner, interleaved in between
                to randomise burst errors (Carlson et al., §13.4).
        \end{itemize}
  \item \textbf{Interleaver / De-interleaver}  
        Block-random interleavers sized to the coded payload
        (indices generated once with a fixed RNG seed).
  \item \textbf{Mapper / Modulator}  
        QPSK and 16-QAM with Gray labelling
        (TAC Topics, §7.7).  Symbols are loaded into an $N\!=\!256$
        sub-carrier OFDM IFFT; a 25 %-cyclic prefix is appended.
  \item \textbf{Channel Model}  
        \begin{itemize}
          \item \textit{AWGN}: additive white Gaussian noise only.  
          \item \textit{Rayleigh flat-fading}: single tap
                $h\!\sim\!\mathcal N_\mathbb C(0,1)$
                plus AWGN (Carlson et al., Fig. 9-8).
        \end{itemize}
        Noise power is set from the desired
        $E_b/N_0$ via
        $N_0 = 1 /\bigl(R_c\log_2M\bigr)$,
        where $R_c$ is the overall code rate.
  \item \textbf{Demodulator \& Decoders}  
        Hard-decision QAM/QPSK demapper, followed by de-interleaving
        and the corresponding FEC decoder(s).
  \item \textbf{Error Counter}  
        Bit errors and frame (re-Tx) errors are counted to produce
        the empirical BER and retransmission probability
        $P_{\text{re}}$.
\end{enumerate}

\begin{figure}[h]
\centering
\begin{tikzpicture}[node distance=6mm,>=latex,font=\small]
  \node[draw,rectangle] (src) {ASCII / Huffman};
  \node[draw,rectangle,right=of src] (fec)  {FEC \\ (Conv, Ham, Both)};
  \node[draw,rectangle,right=of fec] (int)  {Interleaver};
  \node[draw,rectangle,right=of int] (map)  {QPSK / 16-QAM \\ + OFDM};
  \node[draw,rectangle,right=of map] (chan) {AWGN / Rayleigh};
  \node[draw,rectangle,right=of chan] (dmap){Demap \\+ FFT};
  \node[draw,rectangle,right=of dmap] (dint){De-interleaver};
  \node[draw,rectangle,right=of dint] (dfec){FEC Decoder};
  \node[draw,rectangle,right=of dfec] (sink){ASCII / Huffman Decode};
  \draw[->] (src)--(fec)--(int)--(map)--(chan)--(dmap)--(dint)--(dfec)--(sink);
\end{tikzpicture}
\caption{End-to-end block diagram used in Phase II.}
\label{fig:sys_block_diag}
\end{figure}

\end{comment}
%--------------------------------------------------------------------
\subsubsection{Theoretical re-transmission probability}

For a $(15,11)$ Hamming code an ARQ is triggered if the decoder
detects \(\ge 2\) bit errors in the codeword. The probability is Equation \ref{eq:Pret}.

\begin{equation}
      P_{ret} \approx \begin{pmatrix}n \\ 1\end{pmatrix}p(1-p)^{n-1} \text{\cite{Work_TAC2025}}
      \label{eq:Pret}
\end{equation}

This yields Figure \ref{fig:Pret_Teo}.

%--------------------------------------------------------------------
\subsection{Simulation results}

Figures \ref{fig:conv_fig}–\ref{fig:concat_fig} show the measured BER
for all four channel/modulation pairs.  In the block-code and
concatenated plots the dashed lines add the \emph{simulated} re-Tx
probability, which matches the theoretical curve above for
high SNR.

\begin{comment}

\begin{figure}[h]
  \centering
  \includegraphics[width=0.8\linewidth]{%%%conv_fig.pdf}
  \caption{BER for convolutional coding only.}
  \label{fig:conv_fig}
\end{figure}

\begin{figure}[h]
  \centering
  \includegraphics[width=0.8\linewidth]{%%%ham_fig.pdf}
  \caption{BER (\textcolor{black}{solid}) and $P_{\text{re}}$
           (\textcolor{blue}{dashed}) for Hamming
           $(15,11)$ block coding.  The dotted curve is the theoretical
           $P_{\text{re,\,theo}}(P_b)$.}
  \label{fig:ham_fig}
\end{figure}

\begin{figure}[h]
  \centering
  \includegraphics[width=0.8\linewidth]{%%%concat_fig.pdf}
  \caption{BER (\textcolor{black}{solid}) and $P_{\text{re}}$
           (\textcolor{blue}{dashed}) for concatenated
           Hamming + convolutional coding.}
  \label{fig:concat_fig}
\end{figure}

\end{comment}
%--------------------------------------------------------------------
\subsection{Discussion}

\begin{comment}

\begin{itemize}
  \item \textbf{Modulation vs.\ Channel.}
        Consistently with \cite{Work_TAC2025}, QPSK outperforms
        16-QAM by $\approx$4–6 dB in both channels; Rayleigh fading
        raises the BER floor unless diversity or coding gain compensates
        for deep fades (Carlson et al., Fig.\,10-14).
  \item \textbf{Coding gain.}
        The convolutional code shifts the BER curve about
        $%%%<\!G_c\!\text{ dB}>$ to the left (at BER $=10^{-5}$)
        relative to uncoded transmission.  The block code alone yields
        $%%%<\!G_b\!\text{ dB}>$; the concatenated scheme combines both
        gains and meets the lab target of BER $<10^{-5}$ at
        $E_b/N_0 = 9$ dB in AWGN.
  \item \textbf{Retransmissions.}
        The simulated $P_{\text{re}}$ coincides with the theoretical
        curve at high SNR, confirming correct error-detection logic.
        In Rayleigh, $P_{\text{re}}$ flattens above $10^{-4}$ because
        burst errors from deep fades violate the “max 1 error/word”
        assumption; interleaving mitigates but does not eliminate this
        effect.
  \item \textbf{Source coding impact.}
        Huffman reduces the transmitted bit rate from 8 b/char to
        $\approx4.09$ b/char, effectively doubling throughput at the
        same spectral occupancy.  The BER curves are \emph{unchanged}
        because the channel sees random bits either way, but under
        errors Huffman can corrupt more than one character—a trade-off
        discussed in Sec.\,\ref{sec:throughput}.
\end{itemize}
\end{comment}

